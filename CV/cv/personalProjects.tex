%-------------------------------------------------------------------------------
%	SECTION TITLE
%-------------------------------------------------------------------------------
\cvsection{Professional Developements}


%-------------------------------------------------------------------------------
%	CONTENT


\begin{cventries}

%---------------------------------------------------------
  \cventry
    {An online cryptocurrency course} % Degree
    { \href{https://www.coursera.org/learn/cryptocurrency}{Bitcoin and Cryptocurrency Technologies}, \href{https://www.coursera.org/instructor/anarayanan}{Prof. Arvind Narayanan} } % Institution
    {Princeton University, Coursera} % Location
    {2020 - 2021} % Date(s)
    {
      During this course, which was part of the master's program at Princeton University, I learned fundamental concepts regarding Bitcoin and other cryptocurrencies. I learned about how they achieve decentralization(!), how mining is done, alternative consensus protocols, etc. During this course, I implemented a simple blockchain network. 
    }

  \cventry
    {A cryptocurrency powered by a redefined PoA protocol} % Degree
    { \href{https://github.com/Crystaline-Coin/crystaline}{Crystaline} } % Institution
    {University of Tehran} % Location
    {Aug. 2021 - Sep. 2022} % Date(s)
    {
      Developed as a proof of concept on pure Python, this cryptocurrency incorporates a newly defined Proof of Activity as its primary consensus protocol. It was designed and researched by me and several other students of the University of Tehran.
    }

  \cventry
    {The moderator for DM contests of University of Tehran} % Degree
    {\href{https://github.com/kamali-sina/dmcb}{DMCB}} % Institution
    {University of Tehran} % Location
    {Jun. 2022 - Aug. 2022} % Date(s)
    {
      Developed as a moderator for the discrete mathematics course at the University of Tehran. DMCB was created using the Django framework. 
      }

  \cventry
    {A Kaggle clone made by interns at Divar} % Degree
    {\href{https://github.com/kamali-sina/mini-kaggle}{Mini Kaggle}} % Institution
    {Divar} % Location
    {Jun. 2021 - Sep. 2021} % Date(s)
    {
      A Kaggle clone made using the Django framework as a learning project at Divar in the summer of 2021. We used several software developing tools and libraries including Docker, Celery, Pandas, and etc. I learned how to effectively work as a software development team during this time.
    }

  \cventry
    {A text-based game created as a passion project} % Degree
    {\href{https://github.com/kamali-sina/Sins-Virtues-Legacy}{Sins \& Virtues}} % Institution
    {University of Tehran} % Location
    {Feb. 2021 - present} % Date(s)
    {
      Developed as a passion project. At first, it was \href{https://github.com/kamali-sina/Sins-Virtues}{programmed using Python}, but after further consideration, it was rewritten using C++. The game encapsulates a rich set of fun mechanics. Follow this link to \href{https://github.com/kamali-sina/Sins-Virtues-Legacy/blob/main/Tutorial.md}{learn more}.
      }

%---------------------------------------------------------
\end{cventries}
